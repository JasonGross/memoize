\documentclass{article}
\usepackage{geometry}
\pagestyle{empty}

\usepackage{memoize}
\memoizeset{memo dir}

\usepackage{tikz}
\usetikzlibrary{shapes}

\begin{document}

Memoize supports TikZ pictures with {\tt overlay}s and other contructs which
make the typeset material stick out of the bounding box, like \TeX's {\tt
  \string\llap} and {\tt \string\rlap}.  In the pictures below, the bounding
box (the size of the typeset material as perceived by \TeX) is the blue node.

When Memoize externalizes a picture, it does not put it on a PDF page of size
precisely equal to the bounding box of the picture.  By default,\footnote{The
  default is motivated by standard values of {\tt \string\pdfhorigin} and {\tt
    \string\pdfvorigin}, which is one inch.} it pads the picture by one
inch\footnote{For those who know what I'm talking about: the default padding is
  actually one \emph{true} inch.} on each side of the bounding box.  The
picture in Figure~\ref{fig:without-padding} is thus externalized into a page of
size indicated by the red rectangle.

After two compilations of this file, the southern pin of
Figure~\ref{fig:without-padding} will still be entirely visible, while its
eastern pin will lose two characters.\footnote{You will see the entire eastern
  pin after the first pass, but don't let this fool you!  In the first pass,
  Memoize does not typeset the picture by including it from external graphics,
  but uses the box that was typeset normally.  But it the second pass, the
  picture is ``typeset'' by including the externalized graphics, and because
  the externalized graphics sticks out of the PDF page, a part of it will be
  missing.} By now, it should be clear why: the rightmost part of the eastern
pin falls out of the memo PDF page.

\begin{figure}[t]
  \begin{center}
    \begin{tikzpicture}
      \node[align=center,text width=0.4\linewidth,draw,thick,cyan,
        pin={[overlay,align=left,text width=0.3\linewidth]
          east:an {\tt overlay}ed pin},
        pin={[overlay,]south:an {\tt overlay}ed pin}]
      {I want to horizontally center this node on the page,
        so I ignore the pins by {\tt overlay}ing them.};
      % This code shows the part of the picture that actually makes it
      % into .memo.pdf.
      \draw[red,overlay]
      ([shift={(-1in,-1in)}]current bounding box.south west)
      rectangle
      ([shift={(1in,1in)}]current bounding box.north east);
    \end{tikzpicture}
  \end{center}
  \caption{Without {\tt padding}}
  \label{fig:without-padding}
\end{figure}

The solution is very simple, if manual.  To show the entire eastern pin
(Figure~\ref{fig:with-padding}), we need to increase the padding on the right.
1em will do: \textcolor{olive}{\tt \string\memoizeset\{padding right=1in+1em\}}.  The size of
the memo pdf page is shown by the green rectangle.

Note that the default padding on every side is {\tt 1in}, so {\tt padding
  right=1em} would actually decrease the padding and yield a wrong result ---
good thing (simple) math expressions\footnote{The value is interpreted as
  e\TeX's {\tt\string\dimexpr}.} are allowed as a value of the padding keys!

{%
  % This solves our little problem:
  \memoizeset{padding right=1in+1em}% padding keys persist until the end of TeX group
  \begin{figure}[b]
    \begin{center}
      \begin{tikzpicture}
        \node[align=center,text width=0.4\linewidth,draw,thick,cyan,
          pin={[overlay,align=left,text width=0.3\linewidth]
            east:an {\tt overlay}ed pin},
          pin={[overlay,]south:an {\tt overlay}ed pin}]
        {I want to horizontally center this node on the page,
          so I ignore the pins by {\tt overlay}ing them.};
        % This code shows the part of the picture that actually makes it
        % into .memo.pdf.
        \draw[green,overlay]
        ([shift={(-1in,-1in)}]current bounding box.south west)
        rectangle
        ([shift={(1in+1em,1in)}]current bounding box.north east);
      \end{tikzpicture}
    \end{center}
    \caption{With {\tt padding right=1in+1em}}
    \label{fig:with-padding}
  \end{figure}
}

Here is the list of padding keys. And by the way, too much padding can never
hurt.
\begin{itemize}
\item {\tt padding left, padding right, padding top, padding bottom}
\item {\tt padding x, padding y}: the horizontal and the vertical sides
\item {\tt padding}: all sides
\end{itemize}


\end{document}

%%% Local Variables:
%%% mode: latex
%%% TeX-master: t
%%% End:
