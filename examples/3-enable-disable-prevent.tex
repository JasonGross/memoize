\documentclass{article}

% If you load "nomemoize", memoize will not be loaded, but \memoizeset
% etc. will not yield errors.
% \usepackage{nomemoize}

% The order of packages can be important ... memoize must be loaded before
% certain tikz libraries, like "shadows" used by "todonotes".
\usepackage{memoize}

\usepackage{geometry}
\usepackage[linguistics]{forest}
\usepackage{tikz}
\usetikzlibrary{shapes}
\usepackage{todonotes}

% ***Splitting*** (and only splitting, i.e. the second stage of
% externalization) would fail if memoize was loaded here ... bad blood with the
% "shapes" library.  \usepackage{memoize}

\memoizeset{
  memo dir, % don't pollute the main dir
  register=\todo{ O{} +m },  % what is the argument structure of \todo
  prevent=\todo,    % don't memoize within \todo !
  enable=minipage,  % automatically memoize minipages
}


\begin{document}

First of all, {\tt \string\memoizeset\{disable\}} disables all memoization (for
  the local \TeX\ group). And {\tt \string\memoizeset\{enable\}} brings it
back.  There are also short forms {\tt \string\memoizedisable} and {\tt
  \string\memoizeenable}.

{% It is safer to use \memoizedisable within a group than a \memoizedisable --
  % \memoizeenable pair --- what if we want to disable for the entire document?
  \memoizedisable
  
  \begin{center}
    \begin{forest} nice empty nodes%delay={where content={}{coordinate}{}}
      [[[This] [tree]] [[is] [[not] [externalized]]]]
    \end{forest}
  \end{center}
}


We have three keys:
\begin{itemize}
\item {\tt enable}: The given environment or command will be automatically memoized.
\item {\tt prevent}: Within the given environment or command, memoization is
  switched off.
\item {\tt disable}: Disable the effect of {\tt enable} or {\tt prevent} for
  the given environment or command.
\end{itemize}


Externalized TikZ picture can be given either in the {\tt tikzpicture}
environment or by the {\tt\string\tikz}.\todo{But I'm {\tt prevent}ed from
  being externalized, even if I use TikZ behind the scenes!}

\begin{tikzpicture}
  \node[starburst,fill=yellow,draw=red,thick,font=\Huge]{externalized};
\end{tikzpicture}
% The \tikz command has two forms: 
% braced "\tikz{...}" and single-command "\tikz ... ;". Both are supported.
\tikz\node[starburst,fill=yellow,draw=red,thick,font=\Large]{externalized};

% stop automatically memoizing \tikz command --- tikzpicture environment will
% still be automemoized
\memoizeset{
  disable=\tikz,
}

But now we said {\tt disable=\string\tikz}, so the command is not externalized
any more.

\begin{tikzpicture}
  \node[starburst,fill=yellow,draw=red,thick,font=\Huge]{externalized};
\end{tikzpicture}  
\tikz\node[cloud,aspect=5,fill=blue,text=white,thick,font=\large\bf]{not externalized};

By the way, do you know that the pictures on the left above share the same
memo? The same code, the same memo \dots\ {\tt
  01B9F3FF1943DC0B0A1B62261CE9CD5E.memo.pdf}


An externalized minipage:
\fbox{\begin{minipage}[t]{0.6\linewidth}
  A suggestion for an easy control over whether a TikZ picture gets
  externalized or not: if you say
  {\tt\string\memoizeset\{disable=\string\tikz\}}, {\tt tikzpicture} environments
  get externalized, but {\tt\string\tikz} commands don't.  Easy and logical:
  environments are for the big stuff, commands for little jobs.
\end{minipage}}


\end{document}

%%% Local Variables:
%%% mode: latex
%%% TeX-master: t
%%% End:
