\documentclass{article}
\usepackage{docmute}
\enlargethispage{1cm}
\pagestyle{empty}

\usepackage{memoize}
\memoizeset{
  memo filename prefix={chapters/book.memo.dir/}
}

\usepackage{tikz}

\begin{document}

\section{Introduction}

When writing a longer document, it is often desirable to compile only a part of
the document, and there are several ways to achieve that. This example uses the
{\tt docmute} package to achieve that. But what we really want to know here is:
Can we use pictures externalized when compiling the included files when we
compile the main document, and vice versa?  Sure!

The template for names of memo files is not fixed. It is composed of:
\begin{enumerate}
\item the prefix which you can set by memoize key {\tt memo filename prefix}, 
\item the automatically computed md5sum, and
\item the suffix which you can set by memoize key {\tt memo filename suffix}.
\end{enumerate}
Now crucially, the prefix can contain a path (which must exist, in the usual
  setup TeX can't create folders for you). In this document, we have exploited
this to set
\begin{itemize}
\item {\tt memo filename prefix=\{chapters/book.memo.dir/\}} in the main file
  --- note the braces around the value, and the slash at the end! --- and
\item {\tt memo filename prefix=\{book.memo.dir/\}} in the included file, which
  resides in directory {\tt chapters}.
\end{itemize}

\documentclass{article}

\usepackage{memoize}
\memoizeset{
  memo filename prefix={book.memo.dir/},
  readonly      % This can be a good idea when you work on pictures.
}

\usepackage{tikz}

\begin{document}

\section{Included}

The code for this section was input by {\tt \string\input\{chapters/4-chapter\}}.

\begin{flushleft}
  \begin{tikzpicture}[pin distance=4em]
    \node[align=left,text width=0.3\linewidth,draw,dashed,red,
      pin={[align=left,text width=0.4\linewidth]east:
        Delete the memo file ({\tt chapters/<md5sum>.memo.pdf})
        between tests. This will force
        recompilation.}]
    {If you compile the included file (twice) first, the main file will pick up
      this externalized picture.  And vice versa.};
  \end{tikzpicture}
\end{flushleft}

\end{document}


%%% Local Variables:
%%% mode: latex
%%% TeX-master: t
%%% End:


\section{Readonly}

When {\tt \string\memoize\{readonly\}} is in effect, memoize will use any
pictures that were already externalized, but it will not externalize new stuff.
This is great when we work on a picture, because we don't want to see it as an
extra page all the time (and wait for the old version to be split off all the
  time).

One possible workflow is to say {\tt readonly} in the included file, as that is
the file that you will be compiling over and over again when you work on it.
Nothing will get externalized, but the compilation will be fast, because
externalized pictures will be used.  Once you want to look at the whole book,
you compile the main file (which contains no {\tt readonly}) and everything
gets externalized.

\section{Conclusion}

Easy, right?





\end{document}

%%% Local Variables:
%%% mode: latex
%%% TeX-master: t
%%% End:
